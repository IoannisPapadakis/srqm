% Visualizations
  % Bars and dots
  % Pie charts
% Crosstabulation
  % Table frequencies
% Tests
  % Chi-squared test
  % Odds ratios
    % Computation
    % Stata implementation

\documentclass[t]{beamer}
\usetheme{hkllite}

\title{logistic regression}
	\author{François Briatte \& Ivaylo Petev}
	\date{Week~\#11}

\begin{document}
	
    \frame[plain]{
		\titlepage\\[7em]
		\tableofcontents[hideallsubsections]
		}

	%
	%
	
	\begin{frame}[t]{Outline}
	
	\begin{block}{Equations}
	
	$$\text{Linear model:~} Y = \beta_0 + \beta_1 X_1 + \beta_2 X_2 + \epsilon$$
	$$\text{Logistic model:~} Pr(Y=1) = \frac{exp(\beta_0 + \beta_1 X_1 + \beta_2 X_2 + \epsilon)}{1+exp(\beta_0 + \beta_1 X_1 + \beta_2 X_2 + \epsilon)}$$
	\end{block}
		
	\red{Bivariate statistics} look at the probability of an association between two variables and are part of \red{inferential statistics}.\\[.5em]
	
	Before going that far, we need to learn how to \red{crosstabulate} variables with each other, as tables or graphs.\\[.5em]
	
	We will also introduce a few non-parametric tests that do not rely on the normal distribution.\\[.5em]
	
		\begin{columns}[T]
		\column{.35\textwidth}
			\tableofcontents[hideallsubsections]
%			\vspace{1.25em}
%			… and a lot more awesomeness later on with linear correlation and regression modelling.
		\column{.45\textwidth}
			\begin{center}
				\href{http://ideas.repec.org/c/boc/bocode/s447101.html}{\includegraphics[width=\textwidth]{images-old/table.png}}
			\end{center}		
		\end{columns}
		\vspace{2em}
	\end{frame}
		
	\section{Visualizations}

	\begin{frame}[t]{Visualizations}
	
	If your independent variables include \red{categorical variables}, try comparing your dependent variable across their categories:
	
	\begin{itemize}
		\item Some \textbf{ordinal scales} can be treated as categories,\\e.g. high-low income, left-right political position…
		\item Some \textbf{nominal variables} are commonly available,\\e.g.  geographical area, religious beliefs, ethnic groups…
		\item Some \textbf{binary variables} often act as controls in comparisons,\\e.g. gender, democratic/dictatorial, religious/atheist…
	\end{itemize}
	
	Graphing the variables together in a ``two-way graph'' explores the \red{possibility of a relationship} between them.
	\end{frame}
	
	% something is strange: the Protestant ethic is the less greedy
	% while Jews are outrun by Muslisms join at top (think of what
	% means for how antisemitism must work in practice for people)...
%	\fullslide{images-old/comp-dot.pdf}
	% ... and men *seem* to come out as greedier than women.
%	\fullslide{images-old/comp-dot-over.pdf}
	% and here, more extreme-left partisans *seem* reluctant to any
	% form of poor immigration than among right-wing partisans...
%	\fullslide{images-old/comp-hbar-stack.pdf}
	% ... but women on the extreme-left are 5 percentage points
	% less reluctant to poor immigration than extreme-left men.
%	\fullslide{images-old/comp-hbar-stack-over.pdf}
	
	\subsection{Bars and dots}
	
	\begin{frame}[t]{Operationalization: Bars and dots}

	\begin{itemize}
		\item Use \texttt{gr dot} to compare the mean, quantile or proportion of a \red{continuous variable} across groups.
		
		\begin{itemize}
			\item Some interval scales can be treated as pseudo-continuous.
			% e.g. 4-pt ``agree/disagree'' or 10-pt ``left/right'' scales.
			\item Use the \texttt{sort(1)des} option to order the groups by descending order.
			\item Use the \texttt{ylab()} option to label the horizontal axis correctly.
			\item Use the \texttt{exclude0} option if the null value is not meaningful.
		\end{itemize}
		
		\item Use \texttt{gr bar} or \texttt{gr hbar} with the \texttt{per stack} options to compare proportions of an \red{ordinal variable} as percentages across groups.
		
		\begin{itemize}
			\item Use \texttt{tab, gen()} to create dummies of each group category.
			\item Use the \texttt{legend(lab(\# "..."))} option to rewrite the legending.
			\item Use the \texttt{bar(\#,col(...))} option to apply a sensible color scheme.
			\item Use the \texttt{blab(bar, pos(center) format(\%9.2f))} to add labels.
		\end{itemize}
		
		\item Play around with \texttt{over()}, \texttt{by()} and other graph options, looking at course do-files for examples.
	\end{itemize}

	\end{frame}
	
	\subsection{Pie charts}

	\begin{frame}[t]{Operationalization: Pies}

		Technically, you might want to use pie charts with the \texttt{by()} option to look at proportions over categories, as they  do at \textit{The Economist}:\\[1em]

		\href{http://andrewgelman.com/2011/04/attractive_but/}{\includegraphics[width=\textwidth]{images-old/uglyassgraph.jpg}}

	\end{frame}
		
	
	\begin{frame}[t]{A Modest Proposal For Pie Charts (Hmm. Pie.)}

		The problem with pie charts is that the eye reads \red{polar coordinates} \href{http://en.wikipedia.org/wiki/Pie_chart}{substantially less efficiently than it reads cartesian coordinates}.\\[1em]

		\begin{columns}[T]
			\column{.35\textwidth}
			Simple solution:\\
			\red{Do not use pie charts.}\\[4.75em]
			\tiny{(except for rare instances of binary-type proportions shown in exploded slices over a fairly limited number of groups with clear contrasting colours and possibly percentage labels shown outside the plot; \href{http://andrewgelman.com/2011/04/one_more_time-u/}{even the best pie charts from \textit{The Economist} can be redrawn into more readable plots} if you think about it.)}
			\column{.6\textwidth}
			\includegraphics[width=\textwidth]{images-old/pie-problem.jpg}
		\end{columns}

	\end{frame}
	
	% Pie charts are used for marketing, politics 
	% and other forms of obsfucation. ``3-D'' pie charts
	% qualify as an illegal weapon in this course.
	
	
	
	\section{Crosstabulation}

	\begin{frame}[t]{Crosstabulations}
	
	Assume a population composed of 50\% men and 50\% women, with 75\% of the population supporting abortion and 25\% opposing it.\\[1em]

	If men and women hold identical views on abortion, each group will independently reflect the population percentages:\\[1em]
		
%	\begin{center}
%	\textit{Example: gender and abortion}\\[1em]
%	\begin{tabular}{lccc}
%	\toprule
%	& \multicolumn{3}{c}{Views on abortion}	\\
%	\cmidrule(r){2-4}
%	Gender & Support & Oppose & Total\\
%	\cmidrule(r){1-4}
%	Men	  & 75\%	& 25\% 	& 100\%\\
%	Women & 75\%  	& 25\%	& 100\%\\
%	\bottomrule
%	\end{tabular}
%	\end{center}

		
	\end{frame}
	
	\begin{frame}[t]{Crosstabulations}
	
%	\begin{center}\vspace{-1em}
%	\textit{Example: politics and attitudes towards migrants}\\[1em]
%	\begin{tabular}{lcccc}
%	\toprule
%	& \multicolumn{4}{c}{Allow poor migrants}	\\
%	\cmidrule(r){2-5}
%	Political categories & Many & Some	& Few & None\\
%	\cmidrule(r){1-5}
%	Extreme-left	& 42	 & 76 		& 37	 & 12 \\
%	Left	 		& 63  & 284	& 161	 & 46\\
%	Center	 		& 31  & 209	& 238	 & 61\\
%	Right	 		& 17	 & 218	& 234	 & 102\\
%	Extreme-right	& 1	 & 17	 	& 35	 & 27\\
%	\bottomrule
%	\end{tabular}
%	\\
%	\vspace{.5em}
%	Source: European Social Survey 2008, France.\\[1em]
%	\end{center}
	
	Note: unweighted data, since the Chi-squared test is non-parametric (it relies on a distinct distribution).
	
	\end{frame}
	
%	\fullslide{images-old/crosstabs-s6.pdf}
	
	\subsection{Table frequencies}

	\begin{frame}[t]{\red{Observed} cell percentages}
	
%\begin{center}\vspace{-1em}
%\textit{Example: politics and attitudes towards migrants}\\[1em]
%\begin{tabular}{lccccc}
%\toprule
%& \multicolumn{5}{c}{Allow poor migrants}	\\
%\cmidrule(r){2-6}
%Political categories & Many & Some	& Few & None & \textbf{Percentage}\\
%\cmidrule(r){1-6}
%Extreme-left	& 2.20 & 3.98	& 1.94	 & 0.63	 & \textbf{8.74}\\
%Left	 		& 3.30 & 14.86	& 8.42	 & 2.41 & \textbf{28.99}\\
%Center	 		& 1.62 & 10.94	& 12.45  & 3.19 & \textbf{28.21}\\
%Right	 		& 3.46 & 9.63	& 9.99	 & 5.54 & \textbf{28.64}\\
%Extreme-right	& 0.98 & 2.72	& 2.82	 & 1.56 & \textbf{8.10}\\
%\textbf{Percentage}	& \textbf{12.10} & \textbf{33.63} & \textbf{34.90}	& \textbf{19.37} & \textbf{100\%}\\
%\bottomrule
%\end{tabular}
%\end{center}
	
	The \red{cell percentage} of each cell in the table is known by comparing each cell frequency to the whole sample distribution.
	
	\end{frame}
	
	\begin{frame}[t]{\red{Observed} row percentages}
	
%	\begin{center}\vspace{-1em}
%	\textit{Example: politics and attitudes towards migrants}\\[1em]
%	\begin{tabular}{lccccc}
%	\toprule
%	& \multicolumn{5}{c}{Allow poor migrants}	\\
%	\cmidrule(r){2-6}
%	Political categories & Many & Some	& Few & None & \textbf{Total}\\
%	\cmidrule(r){1-6}
%	Extreme-left	& \rd{25.15} & \rd{45.51}	& \rd{22.16} & \rd{7.19} & \textbf{\rd{100\%}}\\
%	Left	 		& \bl{11.37} & \bl{51.26}	& \bl{29.06} & \bl{8.30} & \textbf{\bl{100\%}}\\
%	Center	 		& \rd{5.75} & \rd{38.78}	& \rd{44.16} & \rd{11.32} & \textbf{\rd{100\%}}\\
%	Right	 		& \bl{2.98}	 & \bl{38.18}	& \bl{40.98} & \bl{17.86} & \textbf{\bl{100\%}}\\
%	Extreme-right	& \rd{1.25}	 & \rd{21.25}	& \rd{43.75} & \rd{33.75} & \textbf{\rd{100\%}}\\
%	\bottomrule
%	\end{tabular}
%	\end{center}
	
	The proportion of each row group within each column group is known by reading the \red{row percentages} of the crosstabulation.
	
	\end{frame}

	\begin{frame}[t]{\red{Observed} column percentages}
	
%	\begin{center}\vspace{-1em}
%	\textit{Example: politics and attitudes towards migrants}\\[1em]
%	\begin{tabular}{lcccc}
%	\toprule
%	& \multicolumn{4}{c}{Allow poor migrants}	\\
%	\cmidrule(r){2-5}
%	Political categories & Many & Some	& Few & None\\
%	\cmidrule(r){1-5}
%	Extreme-left	& \rd{27.27} & \bl{9.45}	& \rd{5.25}	 & \bl{4.84}\\
%	Left	 		& \rd{40.91} & \bl{35.32}	& \rd{22.84}	 & \bl{18.55}\\
%	Center	 		& \rd{20.13} & \bl{26.00}	& \rd{33.76}  & \bl{24.60}\\
%	Right	 		& \rd{11.04} & \bl{27.11}	& \rd{33.19}	 & \bl{41.13}\\
%	Extreme-right	& \rd{0.65}	 & \bl{2.11} 	& \rd{4.96}	 & \bl{10.89}\\
%	\textbf{Total}	& \textbf{\rd{100\%}} & \textbf{\bl{100\%}} & \textbf{\rd{100\%}}	& \textbf{\bl{100\%}}\\
%	\bottomrule
%	\end{tabular}
%	\end{center}
	
	The proportion of each column group within each row group is known by reading the \red{column percentages} of the crosstabulation.
	
	\end{frame}

	\begin{frame}[t]{\red{Expected} frequencies}
	
%	\begin{center}\vspace{-1em}
%	\textit{Example: politics and attitudes towards migrants}\\[1em]
%	\begin{tabular}{lcccc}
%	\toprule
%	& \multicolumn{4}{c}{Allow poor migrants}	\\
%	\cmidrule(r){2-5}
%	Political categories & Many & Some	& Few & None\\
%	\cmidrule(r){1-5}
%Extreme-left 	& 13.5 	& 70.3 	& 61.6 	& 21.7\\
%Left		 	& 44.6 	& 233.1 	& 204.4 	& 71.9\\
%Center		 	& 43.4 	& 226.8 	& 198.8 	& 69.9\\
%Right		 	& 46.0 	& 240.2 	& 210.7 	& 74.1\\
%Extreme-right 	& 6.4 	& 33.7 	& 29.5 	& 10.4\\
%	\bottomrule
%	\end{tabular}
%	\end{center}
	
	The \red{expected frequency} of each cell is its row total multiplied by its column total, divided by sample size.

	\end{frame}
	
	\section{Tests}
	
	\subsection{Chi-squared test}

	\begin{frame}[t]{Chi-squared test}
	
	The \red{Chi-squared test} works by adding the deviation between observed frequencies $O_i$ and expected frequencies $E_i$ for each table cell $i$:
	
	$$\chi^2=\sum_{i=1}^{n} \frac{(O_i - E_i)^2}{E_i}$$	
	
	The values taken by $\chi^2$ follow the Chi-squared distribution, which provides a probability level for $H_0: \chi^2=0$ given the degrees of freedom. The test, then, follows a simple logic:
	
	\begin{itemize}
		\item If \red{$Pr(\chi^2=0) < \alpha$}, $H_0$ is rejected: observed frequencies are significantly different from expected ones.
		\item If $Pr(\chi^2=0) > \alpha$, $H_0$ cannot be rejected: observed frequencies are insignificantly different from expected ones.
	\end{itemize}
	
	\end{frame}
	
	% chi2, odds, exact

	\subsection{Odds ratios}

	\begin{frame}[t]{Comparison with \red{odds ratios}}
					
		\begin{quote}
		``Scotland has 13\% redheads, Kabylie has 4\% redheads.''\\
		``Scots are \red{more likely} to be redheads than Kabyles.''
		\end{quote}
		
		Quantify the second statement.

		\begin{itemize}
			\item \textbf{Treat the dependent variable as a binary `success/failure':}\\1 is success (red hair), 0 is failure (other colour).
						
			$$\textsf{Odds of \textit{p}:~}\frac{\textsf{red hair}}{\textsf{other colour}}=\frac{p}{1-p}=\frac{\textsf{success}}{\textsf{failure}}$$
			\vspace{0em}
			
			\item \textbf{Divide the odds in each group to compare across them:} the odds ratio quantifies their comparative likelihood of success.
			
			$$\theta = \frac{\textsf{odds}_{\textsf{Scotland}}}{\textsf{odds}_{\textsf{Kabylie}}}=\frac{\textsf{odds}_1}{\textsf{odds}_2}=\frac{\textsf{success}_1}{\textsf{failure}_1}\times\frac{\textsf{failure}_2}{\textsf{success}_2}$$
						
		\end{itemize}

	\end{frame}
	
	\subsubsection{Computation}

	\begin{frame}[t]{Comparison with \red{odds ratios}}
					
		\begin{quote}
		``Scotland has 13\% redheads, Kabylie has 4\% redheads.''\\
		``Scots are \red{more likely} to be redheads than Kabyles.''
		\end{quote}
		
		Quantify the second statement.
		
		\begin{columns}[t]
		\column{.55\textwidth}
		\vspace{-.5em}
%		\begin{center}
%		\begin{tabular}{lcc}
%			\toprule
%			& \multicolumn{2}{c}{Hair colour} \\
%			\cmidrule(r){2-3}
%			Population & Red & Other \\
%			\midrule
%			Scotland & .13 & $1-.13=.87$\\
%			Kabylie & .04 & $1-.04=.96$ \\
%			\bottomrule
%		\end{tabular}\\[0em]
%	\end{center}
		
	\column{.25\textwidth}
	\vspace{3.275em}	
	$$\hspace{-3em}\theta = \frac{.13}{.87}\times\frac{.96}{.04}\approx\red{3.5}$$
	\end{columns}
	
	\vspace{1.5em}
	Scots are roughly \red{3.5 times more likely} to have red hair than Kabyles. (\href{http://en.wikipedia.org/wiki/Red_hair}{All figures taken from current Wikipedia estimates.})
		
	\end{frame}	

	\subsubsection{Stata implementation}

	\begin{frame}[t]{Stata implementation: odds ratio}

	Are parliamentary regimes \red{more likely} to select female leaders?\\[1em]
	
	% A parliamentary regime will multiply the likelihood of a female leader by 2 (twice more likely).
	\includegraphics[width=\textwidth]{images-old/comp-prop3.pdf}\\[1em]
	
	Parliamentary regimes are twice more likely to select female leaders: the odds are \red{substantially large} and yet \red{statistically insignificant} at $p < .05$. Small sample size might have induced a Type II Error.
	\end{frame}
	
	% 'or' means 'odds ratio' and 'nolog' means linear (non-logarithmic) odds.

	% The 95% CI of the odds ratio includes 1, i.e. identical likelihood: the odds are insignificant.
	
\end{document}
