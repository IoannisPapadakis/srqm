%
% 2
%
\section{Additional data skills}
\label{sec:additional-data-skills}

% 2.1
%
\subsection{Converting datasets}
\label{sec:data-conversion}
\index{Data!Conversion}

% CSV
\label{sec:insheet}
\index{Data!Conversion!From \textsc{csv} files}

Files that use \underline{c}omma-\underline{s}eparated \underline{v}alues (\textsc{csv}) can be imported into Stata if you carefully clean the file first from any additional element that is not part of the data itself.

% XLS
\label{sec:import-excel}
\index{Data!Conversion!From Excel files}

Microsoft Excel files (\textsc{xls}) are written in a proprietary spreadsheet format that can be imported into Stata~12 with the \cmd{import excel} command.

% Fixed format
\label{sec:infix}
\index{Data!Conversion!From fixed format}

Fixed-format data consists of variables that have been separated by a fixed number of spaces. This format is more complex to manipulate, as you have to determine the precise format of the data and run extensive code to name variables and apply labels, sometimes using a `dictionary' file.
    
Details on how to proceed with fixed-format data in Stata appear in the help page of the \cmd{infix} command, and short examples can be retrieved from online Stata tutorials.
    
For an example with the National Health Interview Survey of 2009%
\footnote{\url{http://www.cdc.gov/nchs/nhis/nhis_2009_data_release.htm}}, %
you can try downloading and uncompressing one of the \textsc{ascii} data files for that year%
\footnote{e.g. \url{ftp://ftp.cdc.gov/pub/Health_Statistics/NCHS/Datasets/NHIS/2009/personsx.exe} (provided in self-extracting \textsc{exe} format)}, %
and then apply the sample Stata statements provided by the data teams to see what is involved in producing the final dataset.%
\footnote{e.g. \url{ftp://ftp.cdc.gov/pub/Health_Statistics/NCHS/Program_Code/NHIS/2009/personsx.do}}
    
    If you try to process the example above, you will realize that fixed format data is difficult to manipulate. %
    Unless you have a \emph{lot} of time on your hands to convert and debug the files, do not engage into complex data wrangling with fixed format data.

% coda:

Last, datasets produced for \textsc{sas}, \textsc{spss} or other statistical software can be converted to Stata using the Stat/Transfer software.%
  \footnote{\url{http://www.stattransfer.com/}} %
StataCorp itself recommends the software,%
  \footnote{\url{http://www.stata.com/products/stat-transfer/}} %
and Sciences Po has made it available on its microlab workstations.

A free alternative consists in using the R statistical software with the \texttt{foreign} and \texttt{Hmisc} libraries to import the data in R and then export it to Stata.%
  \footnote{\url{http://www.statmethods.net/input/importingdata.html}} %
There is absolutely no guarantee, however, that the conversion process will be error-free.

In both cases, if you are taking it on yourself to convert the data rather than to trust the data team responsible for the original data files, then you are exposing yourself to conversion errors and will have to perform time-consuming verifications on your unofficial dataset.

For these reasons, your safest option is either to use official data files provided in the Stata proprietary dataset, or to rely on a machine-readable format like \textsc{csv} to be able to open the file from any software.

% 2.2
%
\subsection{Merging datasets}
\label{sec:merge}


%%%% WBOPENDATA
% \section{Accessing World Bank Open Data in Stata}
% http://data.worldbank.org/news/accessing-world-bank-open-data-in-stata

%%%% MERGING COUNTRY REGIONS AND STD NAMES WITH KOUNTRY


% 2.3
%
\subsection{Rescaling variables for indexes}

You can create indexes out of several variables with \emph{identical} ranges by adding or multiplying them, but when the variables have different ranges, you will need to standardize them on a common scale of measurement if you want to ensure that the resulting index takes each component equally into account.

Standardization is common in disciplines like demography and epidemiology, where mortality rates, for instance, have to be age-standardized to compare across time and space. Such operations typically require additional information about the general population.%
  \footnote{For a Stata tutorial, see \url{http://data.princeton.edu/eco572/std.html}}

The simplest standardized scale will consider the minimum value of the variable to be $0$ and its maximum to be $1$, and will express all other values as a fraction of the maximum, therefore creating a $0$--$1$ scale out of the variable.

\label{sec:gtsd01}%
Use the \cmd{std01} command for that purpose. If the command was not installed by the course setup, run this installation command:%
	
  \begin{fullwidth}
	  \begin{docspec}
		  net install \_gstd01,%
			  from(\url{http://web.missouri.edu/~kolenikovs/stata})
	  \end{docspec}  
  \end{fullwidth}
	
The command is an extension to the Stata \cmd{egen} command, and works as shown below, where we merge three \QOG variables on gender equality to a single \texttt{index} variable of range $0$--$3$:%

\begin{docspec}
	use data/qog2011, clear\\
	d wdi\_gris gid\_rfmi gid\_fptw\\
	egen gris01 = std01(wdi\_gris)\\
	egen rfmi01 = std01(gid\_rfmi)\\
	egen fptw01 = std01(gid\_fptw)\\
	gen index = gris01 + rfmi01 + fptw01
\end{docspec}

There is a fair chance, however, that the result of an index will not intuitively convey much information. It might also suffer from the different variability of each variable, regardless of the common scaling. Further diagnostics are generally in order at that stage, the main one being: what have you \emph{really} gained with an index?

Most of the time, making a decisive choice between two or more variables with meaningful units will indeed be a better idea than dropping both of them for a unit-less measure that comes with little guarantee of being balanced and therefore reliable.\footnote{That being said, I guess that the Society of Indexers would disagree: \url{https://rulesofreason.wordpress.com/2012/11/20/the-international-journal-of-indexing/}}