
\section{Working with the course material}


%
%
\subsection{Course datasets}
	\label{sec:data-sources}
	\index{Datasets!Data sources}%
	%
	%

\newthought{This appendix lists the course datasets} in Table~\ref{tbl:data-sources} and briefly documents how they were assembled. All course datasets are provided in Stata 9/10 \ext{.dta} format on an ``as-is'' basis: please reference their original sources and use them for teaching purposes only.%

\newthought{Modifications to the original files} are coded in the \texttt{srqm\_data.ado} script, which is part of the course utilities.%
  \footnote{\url{https://github.com/briatte/srqm/wiki/course-utilities}} %
  The \texttt{srqm\_pkgs.ado} script will add the world maps files \filename{world-c.dta} and \filename{world-d.dta} to the \data folder.%

\newthought{Additional data sources} are listed on the course wiki.%
  \footnote{\url{https://github.com/briatte/srqm/wiki/data}} %
  Please read the warning note at p.~\pageref{external-data-warning} before considering using external data sources for your research project.%

\bigskip
\begin{table}
  \begin{center}
  \footnotesize
  \begin{tabular}{lll}
    \toprule
    Filename & Data & Year(s) \\
    \midrule
    \emph{Teaching datasets:} & & \\
      \quad \texttt{ess0810}  & \ess  & 2008--2010\\
      \quad \texttt{gss0012}  & \gss  & 2000--2012\\
      \quad \texttt{nhis2009} & \nhis & 2000--2009\\
      \quad \texttt{qog2013}  & \qog  & 2009 ± 3 years\\
      \quad \texttt{wvs2000}  & \wvs  & Wave~4, 2000\\
    \midrule
    \emph{World maps:} & & \\
      \quad \texttt{world-c} & \pkg{spmap} dataset &\\
      \quad \texttt{world-d} & \pkg{spmap} dataset &\\
    \bottomrule
  \end{tabular}
  \end{center}
  \label{tbl:data-sources}%
\end{table}


\paragraph{\ess (\texttt{ess2008})}

The \texttt{ess2008} dataset holds Round~4 (2008) of the \ess (\ESS).%
	\footnote{\url{http://ess.nsd.uib.no/ess/round4/}}

\begin{quote}
	The \ess (the \ESS) is an academically-driven social survey designed to chart and explain the interaction between Europe's changing institutions and the attitudes, beliefs and behaviour patterns of its diverse populations.%
	\footnote{\url{http://www.europeansocialsurvey.org}}
\end{quote}

The \ESS dataset should be used with the following survey weights:

\begin{docspec}
	use data/ess2008, clear\\
	svyset [pw = dweight]
\end{docspec}

See the \ESS weighting guide for details.%
  \footnote{\url{http://ess.nsd.uib.no/ess/doc/weighting.pdf}}

The dataset was downloaded from the ESS data server,%
  \footnote{\url{http://nesstar.ess.nsd.uib.no/}} %
  and the codebook was downloaded from the \ESS data website.%
  \footnote{\url{http://ess.nsd.uib.no/}} %
  Check the cumulative dataset for other ESS survey waves.%
  \footnote{\url{http://ess.nsd.uib.no/downloadwizard/}}

\paragraph{\gss (\texttt{gss0012})}

The \texttt{gss0012} dataset holds data from the U.S. \gss (\GSS) for years 2000-2012.

\begin{quote}
	The \GSS contains a standard 'core' of demographic, behavioral, and attitudinal questions, plus topics of special interest. Many of the core questions have remained unchanged since 1972 to facilitate time-trend studies as well as replication of earlier findings.%
	\footnote{\url{http://www3.norc.org/GSS+Website/}}
\end{quote}

The \GSS dataset should be used with the following survey weights:

\begin{docspec}
	use data/gss0012, clear\\
	svyset vpsu [pw = wtssall], strata(vstrat)
\end{docspec}

% link to Pedlow requires fix
% http://tex.stackexchange.com/questions/12230/getting-percent-sign-into-an-url-in-a-footnote#12233

See Appendix~A of the \GSS codebook%
   \footnote{\url{http://publicdata.norc.org:41000/gss/documents//BOOK/GSS_Codebook_AppendixA.pdf}} %
   and the online technical paper ``Calculating Design-Corrected Standard Errors for the General Social Survey, 1988-2010''%
  \footnote{\url{http://publicdata.norc.org:41000/gss/documents//OTHR/GSS\%20design\%20variables.pdf}} %
   by Steven Pedlow for details, especially if you plan to use older survey years for which the sampling and weighting design are different.%

The data are extracted from the \GSS 1972-2012 cumulative cross-sectional dataset (Release 1, March 2013).%
  \footnote{\url{http://www3.norc.org/GSS+Website/Download/STATA+v8.0+Format/}}

\paragraph{\nhis (\texttt{nhis2009})}

The \texttt{nhis2009} dataset holds sample adult data for years 2000--2009 of the U.S. \nhis (\NHIS).

\begin{quote}
	The \nhis (\NHIS) has monitored the health of the nation since 1957. \NHIS data on a broad range of health topics are collected through personal household interviews. For over 50 years, the U.S. Census Bureau has been the data collection agent for the \NHIS. Survey results have been instrumental in providing data to track health status, health care access, and progress toward achieving national health objectives.%
	\footnote{\url{http://www.cdc.gov/nchs/nhis.htm}} 
\end{quote}

The \NHIS dataset should be used with the following survey weights:

\begin{docspec}
    use "data/nhis2009.dta", clear\\
    svyset psu [pw = perweight], strata(strata)
\end{docspec}

See the IHIS/NHIS user notes on variance estimation for more details.%
  \footnote{\url{http://www.ihis.us/ihis/userNotes_variance.shtml}}

The data come from the Integrated Health Interview Series website.%
  \footnote{\url{http://www.ihis.us/}}

\paragraph{\qog (\texttt{qog2013})}

The \texttt{qog2013} dataset holds the \qog (\QOG) Standard cross-sectional dataset in its most recent revision of May~15, 2013. The data are country-level aggregates centered around 2009 $\pm$ 3 years.%

\begin{quote}
	Our research addresses the questions of how to create and maintain high quality government institutions and how the quality of such institutions influences public policy in a broader sense.%
  \footnote{\url{http://www.qog.pol.gu.se/}}%
\end{quote}

The data and codebook come from the \QOG Standard download page.%
   \footnote{\url{http://www.qog.pol.gu.se/data/qogstandarddataset/}}

\paragraph{\wvs (\texttt{wvs2000})}

The \texttt{wvs2000} dataset holds data from Wave~4 (1999-2004) of the \wvs (\WVS).

\begin{quote}
	The \wvs (\WVS) is a worldwide network of social scientists studying changing values and their impact on social and political life. The \WVS in collaboration with EVS (European Values Study) carried out representative national surveys in 97 societies containing almost 90 percent of the world's population. These surveys show pervasive changes in what people want out of life and what they believe. In order to monitor these changes, the EVS/WVS has executed five waves of surveys, from 1981 to 2007.%
	\footnote{\url{http://www.worldvaluessurvey.org/}}
\end{quote}

The \WVS dataset should be used with the following survey weights:

\begin{docspec}
	use data/wvs2000, clear\\
	svyset [pw = s017]
\end{docspec}

See the \WVS weighting guide for details.%
  \footnote{\url{http://www.jdsurvey.net/jds/jdsurveyActualidad.jsp?Idioma=I&SeccionTexto=0405}}

The data come from the official file found at the \WVS website.%
   \footnote{\url{http://www.wvsevsdb.com/}} %
   This version has encoding issues that are used as examples to teach recoding. The cumulative dataset has different variable names and proper variable encoding. More recent data is also currently getting assembled in Wave~6 (2010-2013) of the \wvs.%
  \footnote{\url{http://www.wvs-online.com/}}


%
%
\subsection{Research projects}
  \index{Coursework!Research project!Final paper}%

The research project is the graded coursework component that you will work on with a student partner throughout the entire semester. It consists in regularly submitting a draft paper with its replication script, in the form of a Stata do-file.%

The research projects for this course consist in completing the following steps throughout the semester:%

\begin{itemize}
  \item \textbf{setting up your computer} to follow the course and access the teaching material from the \SRQM folder (Weeks~1--2);%
  \item \textbf{registering a research topic} with a student partner from your class in the course projects list (Weeks~2--3);%
  \item \textbf{exploring the course datasets} to find variables related to your research topic and select some variables of interest (Weeks~3--4);%
  \item \textbf{submitting your first draft} that presents the topic, the data the distributions of the variables under scrutiny (Weeks~4--5);%
  \item \textbf{revising your draft} and resubmitting it with additional significance tests of associations in the data (Weeks~5--8); and%
  \item \textbf{submitting your final paper} in which you model your data with linear or logistic regression (Weeks~9--12).%
\end{itemize}

\newthought{All teaching material} for the research projects, including the projects list, paper templates and example papers, is distributed through Google Documents and administered in class. You might also be sent additional FAQs by email or through the course website before deadlines.%

\newthought{All deadlines are discussed in class.} The deadline for the first draft is generally set to mid-term (Week~6). The deadline for the revised draft is generally set to Week~10. The deadline for the final paper is generally set to one week after the last course session.%

\begin{itemize}
  \item Send your paper as a \PDF file.
  \item Send your code as a Stata do-file.
  \item Both files should be named after your group shortname.
\end{itemize}

