%
% 1.1
%
\section{Course essentials}%
	%
	%
	\label{sec:topic}%
	\index{Methods}%
	%
	%

	\newthought{The approach to social science} that we follow in this course is driven by a dual logic of inquiry: we start with the description of quantitative data, and we end with its analysis through statistical models. The procedures involved in this process are computational and require some technical knowledge of computers and mathematics.%
		%
		%

		%%% They’re someone who can ask and answer questions about and with data.
		%%% http://www.analyticstory.com/hadley-wickham/

		%%% I A key principle in applied statistics is that you should be able to connect between the raw data, your model, your methods, and your conclusions

		%%%%Controlled experiments are the gold standard, but I never do them!

	%%% I (Some) computer scientists' view: we don't need controlled experiments; we can automatically learn from observational data

	%%% I Psychologists' view: each causal question requires its own experiment

	%%% I Observational scientist's version: each causal question requires its own data analysis
	
	%%% Sample surveys (for the problem of extending from sample to population)

	%%% Descriptive observational research (for the problem of modeling complex interactions and response surfaces)

	%%%Political science is largely an empirical discipline. That is, most of us studying politics do so because we are motivated by real world political events, either historical, current, or even events yet to happen. We want to know why these events happen and how to make sense of them. Political science trys to answer these questions in a rigorous way. Data analysis is thus a critical component of political science, serving two important purposes: (1) providing numerical descriptions or summaries of political phenomena, facilitating comparisons across time, countries, states, people, etc; (2) testing theories, models and hypothesis about politics.


% All this is to say that this class involves a little math. Or, as I like to put it, this class is ‘‘techie for fuzzies’’: an introduction to the way political scientists use the tools of statistics to rigorously understand political events. I assume virtually zero mathematical background on your part, either because you didn’t take math in high school or because you’ve forgotten it. I assume no prior background with using computers for data analysis. The classes will have a heavy ‘‘show-and-tell’’ feel to them, where I will use statistical software to do data analysis.

	%%%forecast the effects of interventions, the trajectory of existing trends, and the likely strategies
	
	%%% http://understandingsociety.blogspot.fr/2009/01/predictions.html
	
	% 1.1.1
	%
	\subsection{Quantitative methods}%
		%
		\label{sec:quantitative-methods}
		\index{Quantitative methods}%
		%
		%

	\newthought{Quantitative methods} designate a branch of social science methodology that applies statistical procedures to experimental or observational data in order to produce explanatory models for the complex, recurrent phenomena that affect populations and organizations. These methods can be used to analyze things like attitudes towards highly skilled and low-skilled immigration in the United States,\cite{HainmuellerHiscox:2010a}, the effects of a program to develop fertilizer use in Kenya,\cite{DufloKremer:2009a} or the historical vote that put Adolf Hitler into power in interwar Germany\cite{KingRosen:2008a}.%
		%
		%

	The models produced by quantitative research account for the regularities that exist in the data by estimating how a set of independent, explanatory variables can predict the value of a dependent, outcome variable. To what extent, for example, is the prevalence of HIV/AIDS predictable from the level of economic inequality and degree of political unrest in a country? Does the support for violent action vary with age and education, in what direction, by how much, within what range and at what rate? A quantitative model can estimate these relationships, on top of which researchers develop theories to explain what causal paths are followed in the model.%
		%
		%

	A strong background assumption behind such questions is comparability. In quantitative data, the units of observation, such as individuals or countries, are defined through a set of commensurable characteristics—the variables. The first and perhaps most important requirement of a quantitative model is that you are measuring roughly the same thing among roughly similar units with sufficient reliability. This is far from obvious when you are aggregating, for instance, development statistics, because many countries have very low statistical capacity (among many other issues).\cite{Jerven:2013a}%
		%
		%

	Each variable of a quantitative model is then attached to a concept, like `household income' or `democratic status,' each of which provides an explanatory component for the dependent variable. Measuring the predictive effect of the `independent' variables onto the `dependent' one is therefore a means to measure the respective influence of each explanatory component onto a given phenomenon, which might be a measure of something, like the number of children in a household or the GDP growth rate of a country, or the probability of occurence of something, like abortion or state collapse.%
		%
		%

	\newthought{From a learning perspective,} what you can immediately figure out of the short description above is that quantitative methods require some attention to terminology: `data' are `observations' described by `variables' made of `values', some of which we `predict' from the others through the `estimation' of their `independent' effects. The topic also requires some (really light) exposure to logic and mathematics. Finally, any quantitative analysis requires some knowledge of the practical aspects of empirical research design, such as data collection or sampling.%
		%
		%

	Contrary to what bookshelves of statistics textbooks and horror stories about equations feeding on human flesh might have led you to believe, your learning approach of quantitative analysis should actually have much more to do with its practice than with its theory. The practice of quantitative data analysis implies, for example, that you have to try things out until `they work' at the technical level. This aspect of things requires a lot of independent learning through trial-and-error.%
		%
		%
	
	Another practical dimension of quantitative analysis in the social sciences has to do with the limited degree of precision of any social statistic, which makes issues of statistical significance secondary to issues of measurement accuracy and data availability when it comes to social data. When your unit of analysis is a social one, start thinking in rounded figures: the measures are never more precise than what they are, nor the data more representative than what it can possibly be.%
		%
		\footnote{For that matter, a report that would speak of `1.474\% of the general population' is almost never going to be a credible result with social data, because three-digit precision would indicate spectacularly precise estimation.}%
		%
		%
		
% 1.1.2
%
\subsection{Course outline}%
	\index{Course!Outline}

\newthought{The course} is organised in twelve two-hour sessions that run over a single semester. Its content is structured around three teaching goals:%

\begin{enumerate}
  
  % 1. Statistics

	\label{textbooks}%
	\index{Course!Readings}%
  \item The course covers some \textbf{essential aspects of statistical analysis}, from describing variables to running regression models. %
 	  %
  	This learning objective requires that you read from textbooks that apply fundamental statistical theory to social science research. The primary handbook for this course is \footcite{Urdan:2010a}, one of the shortest and most effective introduction to the course topics. %
	
		Towards the end of the class, we turn to \footcite{FeinsteinThomas:2002d}, a clearly worded introduction to quantitative methods for qualitatively-minded social scientists, for additional clarifications on regression modelling.%
  
		\index{Course!Syllabus}%
  	Both textbooks appear in the course syllabus%
		\footnote{\url{https://github.com/briatte/srqm/blob/master/course/syllabus.pdf?raw=true}} as well as in the final bibliography of this document. Please read the course syllabus in full at that stage, and copy the reading schedule to your agenda.%
		
		Note that the course slides will definitely \emph{not} contain enough material for you to understand the theoretical underpinnings of the methods used in class, and that if you skip the readings, it will inevitably reflect on the general quality of your analysis.%

  % 2. Stata

  \item The course also explains \textbf{how to operate Stata}, a statistical software produced by StataCorp. %
	  %
	  StataCorp is a U.S. company that also publishes journal articles and handbooks to assist users in using their software.%
		%
		\footnote{\url{http://stata.com/}} %
		%
		Some Stata Press handbooks, like the recent one by \footcite{Mitchell:2012a} on applied regression, can be used as companions to this course.%
		%
		\footnote{See an indicative list of references at % 	
			\url{https://github.com/briatte/srqm/wiki/stata}} %
		
		Learning to use Stata requires using a computer for research, not just as a clever typewriter or as a Web terminal; it requires practice with using keyboard shortcuts and managing files. The next pages will explain what specific computer skills you will be working on during the course.%
		
		The baseline advice to survive the computing component of the course is very simple: practice by writing code every week of class. If this is going to be the only time in your student life where you get to write statistical code, make sure that you get the most out of it. There is a fair chance that you will be offered to use that skill one day.%

  % 3. Research

  \item The course finally works assists students to work in pairs over \textbf{small-scale research projects}, on which the grading for the course is based. %
	  %
		The course was conceived with a hands-on focus and is organised around the  writing of an empirical research paper supported by a replication script written in Stata code. Each student pair submits two draft versions of the code and paper during the semester, and one final version at the end of the class.%
		%
		%
		
		As indicated in the course syllabus, you can turn to \footcite{BoothWilliams:2003v} for a detailed explanation on how to prepare a research project if you have no experience in social research, and to \footcite{White:2005a} if you need a refresher on how to write an empirical research paper. All other instructions about the project are covered exclusively in class, which you should attend every week and catch up if absent.%
		%
		\footnote{The course sessions are cumulative: you cannot just skip one as if it had never happened. Your weekly workload \emph{will} extend considerably if you skip one week with the hope of catching up later or at the last minute–both strategies that \emph{never} play out well.}%
		%
		%
		
		You are encouraged to use your research project for other purposes than this course: think of it as work that you will be able to add to your student portfolio.%
		%
		%

\end{enumerate}

	\index{Course!Computer requirements}%
	The course requires that you have access to a recent computer that can connect to the Internet, read/write \PDF files and uncompress \ZIP archives, all of which are native features on most current systems. The course also requires that you can use Google Documents to edit and share your research project files.%
		\footnote{\url{https://docs.google.com/}}%
		%
		%


% 1.1.3
%
\subsection{About this guide}%
	%
	%
	\newthought{This `Stata guide'} was written as an introduction to Stata for students with a background in the social sciences. You are not expected to know anything about statistics, but you are expected to know a few things about social science research from your undergraduate curriculum. You should be familiar, for instance, with notions like cultural capital, gross domestic product per capita and political regimes.%
		%
		%

	We will cover the following topics:%

	\begin{itemize}
		\item This introduction deals with the course basics, essential computer skills and Stata fundamentals. It also explains how to set up a computer for the course.%
	
		\item %
		Section~\ref{ch:data} explains how to prepare data for analysis, and %
		Section~\ref{ch:distr} explains how to visualize distributions. This segment is primarily about description and univariate statistics.%
	
		\item %
		Section~\ref{ch:sig} introduces statistical significance tests, and %
		Section~\ref{ch:ols} introduces ordinary least squares regression. This segment is about association and bivariate statistics.%

		\item %
		Section~\ref{ch:reg} 
		introduces multiple linear regression modelling. %
		It covers a lot of ground in very brief form, which makes the use of a primary textbook particularly necessary at that stage of the course.%

		\item %
		The appendix add some detail about the course material and assignments. The guide also includes an index including all commands cited in the text, and a list of bibliographic references.%
	\end{itemize}

\newthought{Several sections of the guide are still in draft form}, so watch for updates and read it along other documentation. Its writing started with students questions, and several sections were first written as short tutorials concerning specific issues with data management. One thing led to another, and we ended up with the current document. The aim is to cover 90\% of the course by version 1.0.%

	Several students have already provided very valuable feedback on the text—thanks, and keep the feedback flowing in!%